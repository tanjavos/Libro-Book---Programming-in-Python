\section{Conjuntos}\label{conjuntos}

Un conjunto es una colección no ordenada y mutable de objetos únicos. 
Los conjuntos son ampliamente utilizados en lógica y matemática.
Python tambien provee este tipo de datos al igual que otras más convencionales como las listas, tuplas y diccionarios.

Para crear un conjunto especificamos sus elementos entre llaves:

\begin{Verbatim}[frame=single]
>>> s = {1, 2, 3, 4}
\end{Verbatim}

Al igual que otras colecciones, sus miembros pueden ser de diversos tipos:

\begin{Verbatim}[frame=single]
>>> s = {True, 3.14, None, False, "Hola mundo", (1, 2)}
\end{Verbatim}

No obstante, un conjunto no puede incluir objetos mutables como listas, diccionarios, e incluso otros conjuntos.

\begin{Verbatim}[frame=single]
>>> s = {[1, 2]}
Traceback (most recent call last):
  ...
TypeError: unhashable type: 'list'
\end{Verbatim}

No hay que confundir el diccionario vació (\pythoninline{ \{\} }) con un  conjunto vació (\pythoninline{set()}):

\begin{Verbatim}[frame=single]
>>> s = {}
>>> type(s)
  <class 'dict'>
>>> s = set()
>>> type(s)
  <class 'set'>
\end{Verbatim}

Podemos obtener un conjunto a partir de cualquier objeto iterable:

\begin{Verbatim}[frame=single]
s1 = set([1, 2, 3, 4])
s2 = set(range(10))
\end{Verbatim}

Un conjunto puede ser convertido a una lista y viceversa. En este último caso, los elementos duplicados son unificados.

\begin{Verbatim}[frame=single]
>>> list({1, 2, 3, 4})
[1, 2, 3, 4]
>>> set([1, 2, 2, 3, 4])
{1, 2, 3, 4}
\end{Verbatim}



Los conjuntos son objetos mutables. Vía los métodos \pythoninline{add()} y \pythoninline{discard()} podemos añadir y remover un elemento indicándolo como argumento.

\begin{Verbatim}[frame=single]
>>> s = {1, 2, 3, 4}
>>> s.add(5)
>>> s.discard(2)
>>> s
{1, 3, 4, 5}
\end{Verbatim}

Date cuenta que si el elemento pasado como argumento a \pythoninline{discard()} no está dentro del conjunto es simplemente ignorado. En cambio, el método \pythoninline{remove()} opera de forma similar pero en dicho caso lanza la excepción \pythoninline{KeyError}.

Para determinar si un elemento pertenece a un conjunto, utilizamos la palabra reservada \pythoninline{in}.

\begin{Verbatim}[frame=single]
>>> 2 in {1, 2, 3}
True
>>> 4 in {1, 2, 3}
False
\end{Verbatim}

La función \pythoninline{clear()} elimina todos los elementos.

\begin{Verbatim}[frame=single]
>>> s = {1, 2, 3, 4}
>>> s.clear()
>>> s
set()
\end{Verbatim}


El método \pythoninline{pop()} retorna un elemento en forma aleatoria (no podría ser de otra manera ya que los elementos no están ordenados). Así, el siguiente bucle imprime y remueve uno por uno los miembros de un conjunto.

\begin{Verbatim}[frame=single]
while s:
    print(s.pop())
\end{Verbatim}
\pythoninline{remove()} y \pythoninline{pop()} lanzan la excepción \pythoninline{KeyError} cuando un elemento no se encuentra en el conjunto o bien éste está vacío, respectivamente.

Para obtener el número de elementos aplicamos la ya conocida función len():
\begin{Verbatim}[frame=single]
>>> len({1, 2, 3, 4})
4
\end{Verbatim}


\section{Operaciones principales}
Algunas de las propiedades más interesantes de los conjuntos radican en sus operaciones principales: unión, intersección y diferencia.

La unión se realiza con el carácter \pythoninline{|} y retorna un conjunto que contiene los elementos que se encuentran en al menos uno de los dos conjuntos involucrados en la operación.

\begin{Verbatim}[frame=single]
>>> a = {1, 2, 3, 4}
>>> b = {3, 4, 5, 6}
>>> a | b
{1, 2, 3, 4, 5, 6}
\end{Verbatim}

La intersección opera de forma análoga, pero con el operador \pythoninline{&}, y retorna un nuevo conjunto con los elementos que se encuentran en ambos.

\begin{Verbatim}[frame=single]
>>> a & b
{3, 4}
\end{Verbatim}

La diferencia, por último, retorna un nuevo conjunto que contiene los elementos de a que no están en b.

\begin{Verbatim}[frame=single]
>>> a = {1, 2, 3, 4}
>>> b = {2, 3}
>>> a - b
{1, 4}
\end{Verbatim}


Dos conjuntos son iguales si y solo si contienen los mismos elementos (a esto se lo conoce como principio de extensionalidad):

\begin{Verbatim}[frame=single]
>>> {1, 2, 3} == {3, 2, 1}
True
>>> {1, 2, 3} == {4, 5, 6}
False
\end{Verbatim}


Se dice que B es un subconjunto de A cuando todos los elementos de aquél pertenecen también a éste. Python puede determinar esta relación vía el método \pythoninline{issubset()}.

\begin{Verbatim}[frame=single]
>>> a = {1, 2, 3, 4}
>>> b = {2, 3}
>>> b.issubset(a)
True
\end{Verbatim}

Inversamente, se dice que A es un superconjunto de B.

\begin{Verbatim}[frame=single]
>>> a.issuperset(b)
True
\end{Verbatim}

La definición de estas dos relaciones nos lleva a concluir que todo conjunto es al mismo tiempo un subconjunto y un superconjunto de sí mismo.

\begin{Verbatim}[frame=single]
>>> a = {1, 2, 3, 4}
>>> a.issubset(a)
True
>>> a.issuperset(a)
True
\end{Verbatim}

La diferencia simétrica retorna un nuevo conjunto el cual contiene los elementos que pertenecen a alguno de los dos conjuntos que participan en la operación pero no a ambos. Podría entenderse como una unión exclusiva.

\begin{Verbatim}[frame=single]
>>> a = {1, 2, 3, 4}
>>> b = {3, 4, 5, 6}
>>> a.symmetric_difference(b)
{1, 2, 5, 6}
\end{Verbatim}

Dada esta definición, se infiere que es indistinto el orden de los objetos:

\begin{Verbatim}[frame=single]
>>> b.symmetric_difference(a)
{1, 2, 5, 6}
\end{Verbatim}

Por último, se dice que un conjunto es disjoint respecto de otro si no comparten elementos entre sí.

\begin{Verbatim}[frame=single]
>>> a = {1, 2, 3}
>>> b = {3, 4, 5}
>>> c = {5, 6, 7}
>>> a.isdisjoint(b)
False  # No son disjoint ya que comparten el elemento 3.
>>> a.isdisjoint(c)
True   # Son disjoint.
\end{Verbatim}

En otras palabras, dos conjuntos son disjoint si su intersección es el conjunto vacío, por lo que puede ilustrarse de la siguiente forma:

\begin{Verbatim}[frame=single]
>>> def isdisjoint(a, b):
...     return a & b == set()
...
>>> isdisjoint(a, b)
False
>>> isdisjoint(a, c)
True
\end{Verbatim}


