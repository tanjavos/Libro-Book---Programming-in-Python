
\section*{Ejercicios de tipo test}
\addcontentsline{toc}{section}{Ejercicios de tipo test}



\begin{ejercicio}Imagine que definimos la siguiente cadena en Python:

\begin{python}
str = "abcdef"
\end{python}

¿Qué sale cuando tecleamos?
\begin{python}
for i in range(0, len(str), -1):
        print(str[i])
\end{python}

\begin{choices}
\choice 
\begin{verbatim}
a
b
c
d
e
\end{verbatim}
\choice
\begin{verbatim}
abcde
\end{verbatim}
\choice %CORRECT
\begin{verbatim}

\end{verbatim}
\choice
\begin{verbatim}
fedcba
\end{verbatim}
\end{choices}



\end{ejercicio} 

\begin{ejercicio}Imagine que definimos la siguiente cadena en Python:

\begin{python}
str = "abcdef"
\end{python}

¿Qué sale cuando tecleamos?


\begin{python}
for i in range(-4, 2, 1):
        print(str[i])
\end{python}

\begin{choices}
    \choice %CORRECT
\begin{verbatim}
c
d
e
f
a
b
\end{verbatim}
    \choice
\begin{verbatim}
d
c
b
a
\end{verbatim}
    \choice 
\begin{verbatim}

\end{verbatim}
    \choice
\begin{verbatim}
c
d
e
f
\end{verbatim}
\end{choices}

%\solucion{A.}


\end{ejercicio}  

\begin{ejercicio}¿Qué sale cuando ejecutamos el siguiente código?

\begin{python}
i = 0
sum = 0
while i <= 4:
    sum += i
    i = i+1
print(sum)
\end{python}

\begin{choices}
    \choice 10 %CORRECT
    \choice 4
    \choice 5
    \choice 0
\end{choices}

%\solucion{10}



\end{ejercicio}  

\begin{ejercicio}Imagine las siguientes instrucciones en Python:


\begin{python}
n = 5
while n > 0:
    if n == 2:
        break
    n = n - 1
    print(n)
print('Loop ended.')
\end{python}

¿Cual es el resultado?

\begin{choices}
    \choice %CORRECT 
\pythoninline{4}\\
\pythoninline{3}\\
\pythoninline{2}\\
\pythoninline{Loop ended.}
    \choice 
\pythoninline{4}\\
\pythoninline{3}\\
\pythoninline{2}\\
\pythoninline{1}\\
\pythoninline{Loop ended.}
    \choice Es un bucle infinito
    \choice \pythoninline{NameError: name 'break' is not defined}
\end{choices}

\end{ejercicio}  

\newpage

\begin{ejercicio}Imagine las siguientes instrucciones en Python:


\begin{python}
k = 5
i = -4
while (i <= k):
    i = i + 2
    k = k - 2
    print (i+k)
\end{python}

¿Cual es el resultado?

\begin{choices}
    \choice %CORRECT  
\pythoninline{1}\\
\pythoninline{1}\\
\pythoninline{1}
    \choice 
\pythoninline{2}\\
\pythoninline{3}\\
\pythoninline{4}\\
\pythoninline{5}\\
    \choice 
\pythoninline{3}\\
\pythoninline{3}\\
\pythoninline{3}
    \choice Es un bucle infinito

\end{choices}



\end{ejercicio} 

\begin{ejercicio}¿Qué estructuras de control iterativas tiene Python? 

\begin{choices}
    \choice while %choice
    \choice if
    \choice elif
    \choice try
\end{choices}


\end{ejercicio} 

\begin{ejercicio}Imagine el siguiente programa Python:

\begin{python}
acu = int(input("Introduce el valor: "))
cont = int(input("Introduce otro valor: "))
prod = 0
vec = 0
while vec > 0: 
    prod = prod + m
    vec += 1 
print(f"El prod es igual a {prod}")
\end{python}

¿Qué opción es correcta?



\begin{choices}
    \choice prod es una acumuladora y vec una contadora %choice
    \choice vec es una acumuladora y prod una contadora
    \choice acu es una acumuladora y vec una contadora
    \choice acu es una acumuladora y cont una contadora
\end{choices}
\end{ejercicio}  



%%%%%%bucle
\begin{ejercicio} ¿Cuál será el output del siguiente código de Python?
\begin{python}
i = 4
while (i <= 6):
    if i%5 == 0:
        continue
    print(i)
    i += 1
else:
    print("bye")
\end{python}

\begin{choices}
    \choice  Es un bucle infinito   %CORRECT
    \choice   \pythoninline{6 7 bye}
    \choice  \pythoninline{6 bye}
    \choice   \pythoninline{5 6 bye}
\end{choices}

\end{ejercicio}


\begin{ejercicio} ¿Qué imprime por pantalla el siguiente programa?

\begin{python}
x = "abcdef"
i = "i"
while i in x:
    print(i)
\end{python}

\begin{choices}
    \choice no hay resultado   %CORRECT
    \choice \pythoninline{iiiiii}
    \choice \pythoninline{abcdef}
    \choice TypeError
\end{choices}
\end{ejercicio}



\begin{ejercicio} ¿Qué imprime por pantalla el siguiente programa?

\begin{python}
for i in "22":
    for j in "33":
        print(1*i + 2*j, end=" ")
\end{python}


\begin{choices}
    \choice 
\pythoninline{8 8 8 8}

\choice 
\pythoninline{88}

\choice    %CORRECT
\pythoninline{233 233 233 233}
\choice Un error
\end{choices}
\end{ejercicio}

\begin{ejercicio} ¿Cuál de las siguientes es una instrucción de repetición en Python? 

\begin{choices}
    \choice for   %CORRECT
    \choice if
    \choice elif
    \choice try
\end{choices}

\end{ejercicio}


\begin{ejercicio} ¿Qué imprime por pantalla el siguiente código?

\begin{python}
v1 = 6
v2 = 6
while (v1 * v2 >= 0):
    print (v1 * v2)
    v1 = v1 - 3
    v2 = v2 + 3
 \end{python}

\begin{choices}
    \choice 
\pythoninline{27}\\
\pythoninline{0}\\
\pythoninline{-36}

\choice 
\pythoninline{36}\\
\pythoninline{27}

\choice     %CORRECT
\pythoninline{36}\\
\pythoninline{27}\\
\pythoninline{0}

\choice Es un bucle infinito

\end{choices}

\end{ejercicio}

\begin{ejercicio} 
¿Qué imprime por pantalla el siguiente programa?

\begin{python}
c = 3
while (c - 1) > 0:
    if c == 2:
       print(c)
       break
    print(c)
    c = c - 1
print('Loop ended.')
\end{python}

\begin{choices}
    \choice    %CORRECT
\pythoninline{3}\\
\pythoninline{2}\\
\pythoninline{Loop ended.}
    \choice 
\pythoninline{2}\\
\pythoninline{1}\\
\pythoninline{Loop ended.}

    
    \choice 
    \pythoninline{3}\\
\pythoninline{2}\\
\pythoninline{2}\\
\pythoninline{Loop ended.}
     
    \choice NameError: name 'break'  is not defined
\end{choices}
\end{ejercicio}

\newpage


\begin{ejercicio} 
¿Qué imprime por pantalla el siguiente programa?

\begin{python}
mensaje = "Guido van Rossum"
for i in range(2, -2, -3):
        print(mensaje[i])
print("Inventor de Python")
\end{python}

\begin{choices}
    \choice    %CORRECT
\pythoninline{i}\\
\pythoninline{m}\\
\pythoninline{Inventor de Python}
    \choice
\pythoninline{u}\\
\pythoninline{o}\\
\pythoninline{n}\\
\pythoninline{}\\
\pythoninline{Inventor de Python}
    \choice 
\pythoninline{Inventor de Python}
    \choice
\pythoninline{u}\\
\pythoninline{o}\\
\pythoninline{n}\\
\pythoninline{Inventor de Python}
\end{choices}


\end{ejercicio}


\begin{ejercicio} ¿Qué imprime por pantalla el siguiente código?

\begin{python}
n = 2

for i in range(n+1):
    for j in range (i, n, 4):
        print('A', end=".")
        for k in range (j, n, 3):
            print('B', end=",")
            for l in range (k, n - i - j - k):
                print('C', end="-")
\end{python}

\begin{choices}
    \choice 
\pythoninline{A.B,C-C-C-C-A.B,}
    \choice    %CORRECT
\pythoninline{A.B,C-C-A.B,}
    \choice 
\pythoninline{A.B,C-C-A.B}
    \choice 
\pythoninline{A.B,C-C-A.A.B,A.A.A.}
\end{choices}


\end{ejercicio}

\newpage

\begin{ejercicio} ¿Qué imprime por pantalla el siguiente programa?
 \begin{python}
for i in range(1,10):
    if i == 2:
        break
    elif i == 10:
        print("here")
    else:
        print(i)
\end{python}

\begin{choices}
    \choice 1 3 4 5 6 7 8 9 here
    \choice 1   %CORRECT
    \choice 1 3 4 5 6 7 8 9
    \choice 1 here
\end{choices}

\end{ejercicio}


\begin{ejercicio} ¿Qué imprime por pantalla el siguiente programa?

\begin{python}
t = "supercalifragilisticexpialidocious"
for i in range(0, len(t), 8):
	print(t[i], end="")
\end{python}

\begin{choices}
    \choice \pythoninline{sisau}   %CORRECT
    \choice \pythoninline{TypeError}
    \choice \pythoninline{isau}
    \choice \pythoninline{supercal}
\end{choices}

\end{ejercicio}

\begin{ejercicio} ¿Cuál será el output del siguiente código de Python?

\begin{python}
True = False
while(True):
    print("hi", end =" , ")
    break
print("Fin")
\end{python}

\begin{choices}
    \choice An error   %CORRECT
    \choice  \pythoninline{hi}
    \choice  \pythoninline{Fin}
    \choice  \pythoninline{hi , Fin}
\end{choices}

\end{ejercicio}

