\hypertarget{ejercicios}{%
\section*{Ejercicios tipo test}\label{ejercicios.test}}
\addcontentsline{toc}{section}{Ejercicios de tipo test}


\begin{ejercicio} 
\noindent ¿Cuál es la función de la memoria secundaria en un
ordenador?

a) Ejecutar todos los cálculos y la lógica del programa\\
b) Descargar páginas web de Internet\\
c) Almacenar información durante mucho tiempo, incluso después de ciclos
de apagado y encendido\\
d) Recolectar la entrada del usuario
\end{ejercicio}


\begin{ejercicio}
    ¿Cuál de los siguientes contiene ``código máquina''?

a) El intérprete de Python\\
b) El teclado\\
c) El código fuente de Python\\
d) Un documento de un procesador de texto

\end{ejercicio}


\begin{ejercicio} ¿En qué lugar del computador queda almacenada una
variable, como en este caso \verb|x|, después de ejecutar la siguiente
línea de Python?:


\begin{Verbatim}[frame=single]
x = 123
\end{Verbatim}

a) Unidad central de procesamiento\\
b) Memoria Principal\\
c) Memoria Secundaria\\
d) Dispositivos de Entrada\\
e) Dispositivos de Salida
\end{ejercicio}


\begin{ejercicio} ¿Qué mostrará en pantalla el siguiente programa?:

\begin{Verbatim}[frame=single]
x = 43
x = x + 1
print(x)
\end{Verbatim}

a) 43\\
b) 44\\
c) x + 1\\
d) Error, porque x = x + 1 no es posible matemáticamente.
\end{ejercicio}

\begin{ejercicio}¿Cual de la siguientes frases es correcta?

a) En la interpretación se traduce a lenguaje máquina cada instrucción del lenguaje de alto nivel, una a una, en tiempo de ejecución.%CORRECTA

b) En la compilación se traduce a lenguaje máquina cada instrucción del lenguaje de alto nivel, una a una, en tiempo de ejecución.

c) En la interpretación se traducen por medio de un programa todas las instrucciones del lenguaje a lenguaje máquina, previamente a su ejecución.

d) La interpretación y la compilación son dos maneras para traducir un programa escrito en un lenguaje de máquina a un lenguaje de alto nivel.

\end{ejercicio}


\section*{Ejercicios respuesta abierta}\label{ejercicios.abierta}
\addcontentsline{toc}{section}{Ejercicios respuesta abierta}



\begin{ejercicio}¿Qué es un programa?\end{ejercicio}

\begin{ejercicio}¿Cuál es la diferencia entre un compilador y un intérprete?\end{ejercicio}

\begin{ejercicio} Explica cada uno de los siguientes conceptos
usando un ejemplo de una capacidad humana:

(1) Unidad central de
procesamiento, 

(2) Memoria principal, 

(3) Memoria secundaria, 

(4)
Dispositivos de entrada, y 

(5) Dispositivos de salida. 

Por ejemplo,
``¿Cuál sería el equivalente humano de la Unidad central de
procesamiento?''.
\end{ejercicio}

\begin{ejercicio} ¿Cómo puedes corregir un ``Error de sintaxis''?.\end{ejercicio}

\begin{ejercicio} Escriba en español una frase semánticamente
comprensible pero sintácticamente incorrecta. Escriba otra oración que
sea sintácticamente correcta pero que contenga errores semánticos.

\end{ejercicio}
