\section*{Ejercicios de tipo test}
\addcontentsline{toc}{section}{Ejercicios de tipo test} 


Imagine que definimos la siguiente cadena (String) en Python:

\begin{Verbatim}
s = "Hola todo el mundo!"
\end{Verbatim}

\noindent ¿Qué sale cuando tecleamos? ({\color{deepred} \textbf{OJO}: no usa el interpretador de Python para hacer los ejercicios tipo test.. en el examen tampoco lo tendrás.....})\\


\begin{ejercicio}
\begin{Verbatim}
s[-8]
\end{Verbatim}

\begin{choices}
    \choice \verb@'e'@
    \choice \verb@'l'@
    \choice \verb@' '@
    \choice \verb@IndexError@
\end{choices}
\end{ejercicio}
%\solucion{l}

\begin{ejercicio}
\begin{Verbatim}
s[2:6]
\end{Verbatim}

\begin{choices}
    \choice \verb@'ola t'@
    \choice \verb@'la to'@
    \choice \verb@'la t'@
    \choice \verb@'ola '@
\end{choices}
\end{ejercicio}

%\solucion{'la t'}


\begin{ejercicio}
\begin{Verbatim}
s[1:4:2]
\end{Verbatim}

\begin{choices}
    \choice \verb@'oa'@
    \choice \verb@'Hola'@
    \choice \verb@'oat'@
    \choice \verb@'Hl '@
\end{choices}
\end{ejercicio}
%\solucion{'oa'}


\begin{ejercicio}
\begin{Verbatim}
s[:5]
\end{Verbatim}

\begin{choices}
    \choice \verb@'ola t'@
    \choice \verb@'Hola '@
    \choice \verb@'tld'@
    \choice \verb@'Htem '@
\end{choices}
\end{ejercicio}
%\solucion{'Hola '}

\begin{ejercicio}
\begin{Verbatim}
s[10:5:-1]
\end{Verbatim}


\begin{choices}
    \choice \verb@'e odo'@
    \choice \verb@'le od@
    \choice \verb@'t alo'@
    \choice \verb@IndexError@
\end{choices}
\end{ejercicio}

%\solucion{'e odo'}

\begin{ejercicio}
\begin{Verbatim}
s + ' Hi!'
\end{Verbatim}

\begin{choices}
    \choice \verb@'Hola todo el mundo! Hi!'@
    \choice \verb@'str +  Hi!'@
    \choice \verb@TypeError: can only concatenate str (not "int") to str@
    \choice \verb@False@
\end{choices}
\end{ejercicio}
%\solucion{'Hola todo el mundo! Hi!'}

\begin{ejercicio}
\begin{Verbatim}
s[len(s) - 4]
\end{Verbatim}

\begin{choices}
    \choice \verb@'n'@
    \choice \verb@'u'@
    \choice \verb@TypeError@
    \choice \verb@IndexError@
\end{choices}
\end{ejercicio}
%\solucion{'n'}

\begin{ejercicio}
\begin{Verbatim}
s[-1:8:-1]
\end{Verbatim}

\begin{choices}
    \choice \verb@'ola tod'@
    \choice \verb@''@
    \choice \verb@'!odnum le '@
    \choice \verb@IndexError@
\end{choices}
\end{ejercicio}
%\solucion{'!odnum le '}

\begin{ejercicio}
\begin{Verbatim}
s[-1:8:-2]
\end{Verbatim}

\begin{choices}
    \choice \verb@'!du e'@
    \choice \verb@'oatd'@
    \choice \verb@''@
    \choice \verb@IndexError@
\end{choices}
\end{ejercicio}
%\solucion{'!du e'}

\begin{ejercicio}
\begin{Verbatim}
s[len(s)-1:8:-2]
\end{Verbatim}

\begin{choices}
    \choice \verb@'!du e'@
    \choice \verb@''@
    \choice \verb@TypeError@
    \choice \verb@IndexError@
\end{choices}
\end{ejercicio}
%\solucion{'!du e'}

\begin{ejercicio}Que valor tiene la siguiente expresión:

1 + 2 ** 3 * 4

\begin{choices}
    \choice 36
    \choice 4097
    \choice 33 %CORRECT
    \choice 108
\end{choices}
\end{ejercicio}
%\solucion{C}

\begin{ejercicio}Considera la siguiente instrucción en Python:

\begin{python}
x = a + 5 - b
\end{python}

a y b son 


a + 5 - b es 

\begin{choices}
    \choice operandos y una expresión %CORRECT
    \choice operadores y instrucción
    \choice instrucciones y condición
    \choice operandos y equación
\end{choices}
\end{ejercicio}
\begin{ejercicio}¿Qué es el resultado de la expresión \texttt{22\ \%3}?

\begin{choices}
    \choice 7
    \choice 1 %CORRECT
    \choice 0
    \choice error
\end{choices}
\end{ejercicio}
\begin{ejercicio}Imagine la siguiente expresión en Python:

\begin{python}
6 * '=' + 3 * '(O)' + 6 * '='
\end{python}

¿Cual es el resultado?


\begin{choices}
    \choice \verb|'======(O)(O)(O)======'| %CORRECT
    \choice \verb|15|
    \choice \verb|False|
    \choice \verb|'6=3(0)6='|
\end{choices}
\end{ejercicio}
\begin{ejercicio}Imagine la siguiente instrucción en Python:

\begin{python}
print('%d tiene %.2d hijos' % ("Luis", 4))
\end{python}

¿Cual es el resultado?

\begin{choices}
    \choice TypeError %CORRECT
    \choice \pythoninline{Luis tiene 04 hijos}
    \choice \pythoninline{Luis tiene 4 hijos}
    \choice \pythoninline{Luis tiene 4.00 hijos}
\end{choices}
\end{ejercicio}
\begin{ejercicio}Imagine la siguiente instrucción en Python:

\begin{python}
print('{1:b} perros, {2} pajaros y {3} gatos'.format(3,4,5,6))
\end{python}

¿Cual es el resultado?


\begin{choices}
     \choice   \pythoninline{100 perros, 5 pajaros y 6 gatos} %CORRECT
    \choice \pythoninline{3 perros, 4 pajaros y 5 gatos}
    \choice \pythoninline{4 perros, 5 pajaros y 6 gatos}
    \choice IndexError
\end{choices}
\end{ejercicio}
\begin{ejercicio}¿Qué es el resultado de la expresión: \verb|3*1**3|?

\begin{choices}
    \choice 3 %CORRECT
    \choice 9
    \choice 1
    \choice 27
\end{choices}
\end{ejercicio}

\begin{ejercicio}¿Cuál será el output del siguiente fragmento de código de Python?

\begin{python}
'%f %s %d you' %(1, 'hello', 4.0)
\end{python}

\begin{choices}
    \choice \pythoninline{'1.000000 hello 4 you'} %CORRECT
    \choice \pythoninline{'1 hello you 4.0'}
    \choice \pythoninline{'1 hello 4.0 you'}
    \choice \pythoninline{'1.0 hello 4.0 you'}
\end{choices}
\end{ejercicio}


\begin{ejercicio}
Imagine la siguiente expresión en Python:

\begin{python}
print(0 * '1' + 2 * '0' + 3 * '4' + "{0:04d}".format(2*11))
\end{python}

¿Cuál es el resultado?


\begin{choices}
    \choice \pythoninline{004440022} %CORRECT
    \choice \pythoninline{0120340022}
    \choice \pythoninline{ValueError:}
    \choice \pythoninline{0 * 1 + 2 * 0 + 3 * 4 + 22}
\end{choices}

\end{ejercicio}

\begin{ejercicio} ¿Cuál será el output del siguiente código de Python?

\begin{python}
'ftup'[int(bool('spam'))]
\end{python}

\begin{choices}
    \choice \pythoninline{t}   %CORRECT
    \choice  \pythoninline{f}
    \choice  Ningun output
    \choice  An error
\end{choices}
\end{ejercicio}



\begin{ejercicio} Imagina que ejecutamos las siguientes instrucciones en la consola:

\begin{python}
>>> str = "examen de python"
>>> str[-1:9:-1][::-1]
\end{python}

¿Cuál es el resultado?


\begin{choices}
    \choice \pythoninline{python}   %CORRECT
    \choice \pythoninline{nohtyp}
   \choice \pythoninline{examen}
    \choice \pythoninline{eaeeyo}
\end{choices}
\end{ejercicio}
\newpage

\begin{ejercicio} ¿Qué imprime por pantalla el siguiente programa?
 
\begin{python}
s = "examen de python"
print(s + ' Hi!')
\end{python}

\begin{choices}
    \choice \pythoninline{examen de python Hi!}   %CORRECT
    \choice \pythoninline{s +  Hi!}
   \choice TypeError: can only concatenate str (not int) to str
    \choice \pythoninline{False}
\end{choices}


\end{ejercicio}


\begin{ejercicio} ¿Cuál de las siguientes afirmaciones sobre variables es FALSA?:

\begin{choices}
 \choice el primer carácter del nombre no puede ser una letra mayúscula   %CORRECT
 
\choice nombre de una variable es una secuencia de letras, dígitos y subrayados

\choice el primer carácter del nombre no puede ser un dígito

\choice  programas utilizan variables para almacenar valores
 
 \end{choices}     



\end{ejercicio}


\begin{ejercicio} 
¿Cuál de las siguientes expresiones NO invierte el string  \pythoninline{s="OKIDOKI"}?

\begin{choices}
    \choice  \pythoninline{s[0:len(s):-1]}   %CORRECT
    \choice  \pythoninline{s[len(s)//2:len(s)][::-1] + s[0:len(s)//2][::-1]}
    \choice  \pythoninline{s[-1]+s[-2:0:-1]+s[0]}
    \choice  \pythoninline{s[::-1]}
\end{choices}
\end{ejercicio}


\begin{ejercicio} Imagina que ejecutamos las siguientes instrucciones en la consola:

\begin{python}
>>> st = "dia del examen python"
>>> t = st
>>> t[len(st) - 4]
\end{python}

¿Cuál es el resultado?

\begin{choices}
    \choice \pythoninline{t}   %CORRECT
    \choice \pythoninline{e}
    \choice \pythoninline{d}
    \choice \pythoninline{ }
\end{choices}
\end{ejercicio}


\begin{ejercicio} ¿Cuál de las siguientes NO es una palabra reservada en Python?:

\begin{choices}
 \choice out   %CORRECT
 
\choice break

\choice import

\choice  in
 
 \end{choices}     


\end{ejercicio}



\begin{ejercicio} ¿Cuál de las siguientes expresiones booleanas NO tiene el mismo resultado que el resto?
\begin{choices}
    \choice \pythoninline{not(-6>10 or -6==10) and (7<=12)}    %CORRECT
    \choice  \pythoninline{not(-6<0 or -6>10) and (7<=12)}
    \choice \pythoninline{-6>=0 and -6<=10 or 7>12}
    \choice \pythoninline{not(-6<10 or -6==10 and 7<=12) }
\end{choices}

\end{ejercicio}

\begin{ejercicio} ¿Cuál es el resultado de la expresión: \verb|4+3**1**5+3|?

\begin{choices}
    \choice 10   %CORRECT
    \choice 7
    \choice 250
    \choice 16810
\end{choices}

\end{ejercicio}


\begin{ejercicio} ¿Cuál es el resultado de la expresión \verb|42 % 11 + 1|?

\begin{choices}
    \choice 6
    \choice 10   %CORRECT
    \choice 3
    \choice 4
\end{choices}

\end{ejercicio}


\begin{ejercicio} 
Ejecutamos las siguientes instrucciones en la consola:
\begin{verbatim}
>>> s = "I love Python!!"
>>> s[-1:-7:-3]
\end{verbatim}

¿Cuál NO devuelve el mismo valor?

\begin{choices}
    \choice \verb|>>> s[-1:-8:-3]|
    \choice \verb|>>> s[-2:0:-5]|   %CORRECT
    \choice \verb|>>> s[:12:-3]|
    \choice \verb|>>> s[:10:-1][0:7:3]|
\end{choices}
\end{ejercicio}


\begin{ejercicio} ¿Qué imprime por pantalla el siguiente programa?

\begin{python}
print('{0:} y {1:4b} y {2:.4f} y'.format("Grados", 3, 45.45))
print('{3} y {4}'.format(5,5,5,5))
\end{python}

\begin{choices}
    \choice  
\pythoninline{Grados y} $\;\;\;\;\;$\pythoninline{11 y 45.4500 y}\\   %CORRECT
IndexError
    \choice 
\pythoninline{Grados y 0003 y 45.4500 y}\\
\pythoninline{5 y 5}
    \choice Unknown format specifier b
    \choice 
\pythoninline{Grados y} $\;\;\;\;\;$\pythoninline{3 y 45.4500 y}\\
\pythoninline{3 y 4}
\end{choices}

\end{ejercicio}


\begin{ejercicio} ¿Qué imprime por pantalla el siguiente código?

\begin{python}
print('Ven a %str para %str de %04d' % ("Valencia", "Fallas", 2022))
\end{python}

¿Cuál es el resultado si la ejecutamos?

\begin{choices}
    \choice \pythoninline{Ven a Valenciatr para Fallastr de 2022}   %CORRECT
    \choice \pythoninline{Ven a Valencia para Fallas de 2022}
    \choice ValueError: unsupported format character 'str' (0x62) at index 7
    \choice \pythoninline{Ven a Valencia para Fallas de 00002022}
\end{choices}

\end{ejercicio}


\begin{ejercicio} ¿Cuál de las siguientes instrucciones hace conversión \textbf{implícita} de tipos?:

\begin{choices}
 \choice \pythoninline{25 / 3.3 * 4}   %CORRECT
 
\choice \pythoninline{int("345")}

\choice \pythoninline{x = 45.7}

\choice  \pythoninline{str(2.4)}
 
 \end{choices}    

\end{ejercicio}

