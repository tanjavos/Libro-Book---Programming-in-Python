\hypertarget{ejercicios}{%
\section*{Ejercicios de respuesta abierta}\label{ejercicios}}
\addcontentsline{toc}{section}{Ejercicios de respuesta abierta}



\begin{ejercicio}Dada la siguiente expresión en Python 

\pythoninline{7 + 5 * 4 / 3 + 5}

¿Cuál es su resultado?

Cámbialo utilizando paréntesis para que su resultado sea 6
\end{ejercicio}

\respuesta{
18,66\\
((7 + 5) * 4) /(3+5)
}



\begin{ejercicio}Escribe una instrucción Python que asigne a una variable la media aritmética de las variables a, b y c (variables de tipo entero). 
\end{ejercicio}
\respuesta{
\pythoninline{media = (a + b + c) / 3}
}

\begin{ejercicio}Escribe las instrucciones Python que permiten calcular sobre 2 variables \pythoninline{días} y \pythoninline{horas} el número de días y horas a los que equivale una cierta cantidad de horas (totalHoras). Por ejemplo, 

- si totalHoras vale 60, entonces días valdrá 2 y horas 12; 

- si totalHoras vale 25, entonces dias valdrá 1 y horas 1 también.
\end{ejercicio}
%\begin{pythonrespuesta}
%dias = totalHoras // 24
%horas = totalHoras   24
%\end{pythonrespuesta}


\begin{ejercicio}Asume que tenemos las siguientes instrucciones de asignación:
\begin{python}
    ancho = 17
    alto = 12.0
\end{python}

Para cada una de las expresiones siguientes, escribe el valor de la
expresión y el tipo (del valor de la expresión).

(a)  \pythoninline{ancho/2}

(b)   \pythoninline{ancho/2.0}

(c)   \pythoninline{alto/3}

(d)   \pythoninline{alto * 5}

(e)   \pythoninline{ancho * 4}
\end{ejercicio}
\respuesta{
a.  \pythoninline{8.5 - float}

b.  \pythoninline{8.5 - float}

c.  \pythoninline{4.0 - float}

d.  \pythoninline{60.0 - float}

e.  \pythoninline{68 - int}
}


\begin{ejercicio}Escribe un programa en Python que usa input para pedirle al usuario su nombre y luego darle la bienvenida. Algo como:\\


\begin{Verbatim}[frame=single, label={\em ejemplos y posibles test de ejecución}]
>>> %Run
  Introduzca su nombre: Tanja
  Hola Tanja!
\end{Verbatim}
\end{ejercicio}


\begin{ejercicio}Escribe un programa en Python que lea del teclado una cantidad en millas como números entero y muestre por pantalla su equivalente en kilómetros. Téngase en cuenta que 1 milla son 1,609344 kilómetros.\\

\begin{Verbatim}[frame=single, label={\em ejemplos y posibles test de ejecución}]
>>> %Run
  ¿Cuantas millas? 200
  en km es 321.8688
\end{Verbatim}
\end{ejercicio}



\begin{ejercicio}Dadas dos variables \verb+a+ y \verb+b+,  
realizar un programa en Python que permita al usuario introducir
dos valores en las mismas, intercambie sus valores y los muestre por 
pantalla. La ejecución del programa debe dar lugar a lo siguiente:\\
\begin{Verbatim}[frame=single, label={\em ejemplo de ejecución}]
>>> %Run 
  Introduce el valor de la variable a: 4
  Introduce el valor de la variable b: 2
  El valor de a es 2
  El valor de b es 4
\end{Verbatim}
suponiendo que 4 y 2 son los valores introducidos por el usuario.
Esto debe funcionar para cualquier par de valores introducidos por el usuario.

Ejecute pruebas a través de la consola y verifique la salida. ¿Tu programa funciona con números negativos? ¿Funciona con letras? ¿Funciona con números reales? ¿Las variables \verb+a+ y \verb+b+ pueden tener tipos diferentes? ¿Debería su programa funcionar para todos estos casos?
\end{ejercicio}



\begin{ejercicio}Escribe un programa en Python que lea del teclado la cantidad de perros que están en un parque. Utiliza 3 variables para calcular la cantidad de cabezas, patas, y orejas hay en el parque. Imprime el resultado en la pantalla.\\

\begin{Verbatim}[frame=single, label={\em ejemplos y posibles test de ejecución}]
>>> %Run
  ¿Cuantas perros hay? 15
  Hay 15 cabezas, 60 patas, y 30 orejas.
>>> %Run
  ¿Cuantas perros hay? 1
  Hay 1 cabezas, 4 patas, y 2 orejas.
\end{Verbatim}
\end{ejercicio}


\begin{ejercicio}Escribe un programa en Python que pide al usuario la cantidad de horas trabajado y la tarifa por hora. Devuelve el salario bruto.\\

\begin{Verbatim}[frame=single, label={\em ejemplos y posibles test de ejecución}]
>>> %Run
  ¿Cuantas horas? 10
  ¿Cual es la tarifa bruto por hora?65
  Salario bruto es: 650.0
>>> %Run
  ¿Cuantas horas? 13.5
  ¿Cual es la tarifa bruto por hora? 10
  Salario bruto es: 135.0
\end{Verbatim}
\end{ejercicio}


\begin{ejercicio}Implementa un programa que calcule la temperatura en grados centígrados a partir de la temperatura en grados Fahrenheit. La fórmula es la siguiente: 
\begin{displaymath}
  C = \frac{5}{9}(F-32)
\end{displaymath} 
La entrada del programa son los grados Fahrenheit introducidos por el usuario. Este valor lo guardaremos en una variable, por ejemplo \verb+F+. Después, nuestro programa calcula la expresión dada por la fórmula y almacena el resultado en otra variable, por ejemplo \verb+C+. El ultimo paso consistirá en imprimir el resultado para el usuario.\\

\begin{Verbatim}[frame=single, label={\em ejemplo de ejecución}]
>>> %Run 
  Introduce los grados Fahrenheit: 84
  84.0 grados Fahrenheit son 28.9 grados Celsius
\end{Verbatim}

Haz más tests de tu programa para verificar sus respuestas utilizando este convertidor online:

\url{https://www.metric-conversions.org/es/temperatura/fahrenheit-a-celsius.htm}

%TODO
%https://www.geeksforgeeks.org/program-distance-two-points-earth/

\end{ejercicio}

