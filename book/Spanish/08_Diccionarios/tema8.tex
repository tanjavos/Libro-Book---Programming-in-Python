\hypertarget{diccionarios}{%
\section{Diccionarios}\label{diccionarios}}

\index{diccionario} \index{diccionario}

\index{tipo!dict} \index{clave} \index{clave-valor par} \index{índice}
\index{}

Un \emph{diccionario} es como una lista, pero más general. En una lista,
los índices de posiciones tienen que ser enteros; en un diccionario, los
índices pueden ser (casi) cualquier tipo.

Puedes pensar en un diccionario como una asociación entre un conjunto de
índices (que son llamados \emph{claves}) y un conjunto de valores. Cada
clave apunta a un valor. La asociación de una clave y un valor es
llamada \emph{par clave-valor} o a veces \emph{elemento}.

Como ejemplo, vamos a construir un diccionario que asocia palabras de
Inglés a Español, así que todas las claves y los valores son cadenas.

La función \texttt{dict} crea un nuevo diccionario sin elementos. Debido
a que \texttt{dict} es el nombre de una función interna, deberías evitar
usarlo como un nombre de variable.

\index{función dict} \index{dict!función}


\begin{Verbatim}[frame=single]
>>> eng2sp = dict()
>>> print(eng2sp)
  {}
\end{Verbatim}


Las llaves, \texttt{\{\}}, representan un diccionario vacío. Para
agregar elementos a un diccionario, puedes utilizar corchetes:

\index{corchetes} \index{corchetes}


\begin{Verbatim}[frame=single]
>>> eng2sp['one'] = 'uno'
\end{Verbatim}


Esta línea crea un elemento asociando a la clave
\texttt{\textquotesingle{}one\textquotesingle{}} el valor ``uno''. Si
imprimimos el diccionario de nuevo, vamos a ver un par clave-valor con
dos puntos entre la clave y el valor:


\begin{Verbatim}[frame=single]
>>> print(eng2sp)
  {'one': 'uno'}
\end{Verbatim}


Este formato de salida es también un formato de entrada. Por ejemplo,
puedes crear un nuevo diccionario con tres elementos. Pero si imprimes
\texttt{eng2sp}, te vas a sorprender:


\begin{Verbatim}[frame=single]
>>> eng2sp = {'one': 'uno', 'two': 'dos', 'three': 'tres'}
>>> print(eng2sp)
  {'one': 'uno', 'three': 'tres', 'two': 'dos'}
\end{Verbatim}


El orden de los pares clave-elemento no es el mismo. De hecho, si tu
escribes este mismo ejemplo en tu computadora, podrías obtener un
resultado diferente. En general, el orden de los elementos en un
diccionario es impredecible.

Pero ese no es un problema porque los elementos de un diccionario nunca
son indexados con índices enteros. En vez de eso, utilizas las claves
para encontrar los valores correspondientes:


\begin{Verbatim}[frame=single]
>>> print(eng2sp['two'])
  'dos'
\end{Verbatim}


La clave \texttt{\textquotesingle{}two\textquotesingle{}} siempre se
asocia al valor ``dos'', así que el orden de los elementos no importa.

Si la clave no está en el diccionario, obtendrás una excepción
(\texttt{KeyError}):

\begin{Verbatim}[frame=single]
>>> print(eng2sp['four'])
  KeyError: 'four'
\end{Verbatim}


La función \texttt{len} funciona en diccionarios y devuelve el número
de pares clave-valor:

\index{función len} \index{len!función}


\begin{Verbatim}[frame=single]
>>> len(eng2sp)
  3
\end{Verbatim}


El operador \texttt{in} funciona en diccionarios; éste te dice si algo
aparece como una \emph{clave} en el diccionario (aparecer como valor no
es suficiente).

\index{miembro!diccionario} \index{operador in} \index{in!operador}


\begin{Verbatim}[frame=single]
>>> 'one' in eng2sp
  True
>>> 'uno' in eng2sp
  False
\end{Verbatim}


Para ver si algo aparece como valor en un diccionario, puedes usar el
método \texttt{values}, el cual retorna los valores como una lista, y
después puedes usar el operador \texttt{in}:

\index{método valores} \index{valores!método}


\begin{Verbatim}[frame=single]
>>> vals = list(eng2sp.values())
>>> 'uno' in vals
  True
\end{Verbatim}


El operador \texttt{in} utiliza diferentes algoritmos para listas y
diccionarios. Para listas, utiliza un algoritmo de búsqueda lineal.
Conforme la lista se vuelve más grande, el tiempo de búsqueda se vuelve
más largo en proporción al tamaño de la lista. Para diccionarios, Python
utiliza un algoritmo llamado \emph{tabla hash} (hash table, en inglés)
que tiene una propiedad importante: el operador \texttt{in} toma la
misma cantidad de tiempo sin importar cuántos elementos haya en el
diccionario. No vamos a explicar porqué las funciones hash son tan
mágicas, pero puedes leer más al respecto en
\href{https://es.wikipedia.org/wiki/Tabla_hash}{es.wikipedia.org/wiki/Tabla\_hash}.

\index{tabla hash} \index{establecer miembro} \index{miembro!establecer}


\begin{exercise}
En Poliformat encuentras una copia del archivo \texttt{words.txt}
Escribe un programa que lee las palabras de \emph{words.txt} y
las almacena como claves en un diccionario. No importa qué valores
tenga. Luego puedes utilizar el operador \texttt{in} como una forma
rápida de revisar si una palabras está en el diccionario.
\end{exercise}


\hypertarget{diccionario-como-un-conjunto-de-contadores}{%
\section{Diccionario como un conjunto de
contadores}\label{diccionario-como-un-conjunto-de-contadores}}

\index{contador}

Supongamos que recibes una cadena y quieres contar cuántas veces aparece
cada letra. Hay varias formas en que puedes hacerlo:

\begin{enumerate}
\def\labelenumi{\arabic{enumi}.}
\item
  Puedes crear 26 variables, una por cada letra del alfabeto. Luego
  puedes recorrer la cadena, y para cada caracter, incrementar el
  contador correspondiente, probablemente utilizando varios
  condicionales.
\item
  Puedes crear una lista con 26 elementos. Después podrías convertir
  cada caracter en un número (usando la función interna \texttt{ord}),
  usar el número como índice dentro de la lista, e incrementar el
  contador correspondiente.
\item
  Puedes crear un diccionario con caracteres como claves y contadores
  como los valores correspondientes. La primera vez que encuentres un
  caracter, agregarías un elemento al diccionario. Después de eso
  incrementarías el valor del elemento existente.
\end{enumerate}

Cada una de esas opciones hace la misma operación computacional, pero
cada una de ellas implementa esa operación en forma diferente.

\index{implementación}

Una \emph{implementación} es una forma de llevar a cabo una operación
computacional; algunas implementaciones son mejores que otras. Por
ejemplo, una ventaja de la implementación del diccionario es que no
tenemos que saber con antelación qué letras aparecen en la cadena y
solamente necesitamos espacio para las letras que sí aparecen.

Aquí está un ejemplo de como se vería ese código:


\begin{python}[frame=single]
palabra = 'brontosaurio'
d = dict()
for c in palabra:
    if c not in d:
        d[c] = 1
    else:
        d[c] = d[c] + 1
print(d)
\end{python}


Realmente estamos calculando un \emph{histograma}, el cual es un término
estadístico para un conjunto de contadores (o frecuencias).

\index{histograma} \index{frecuencia} \index{recorrido}

El bucle \texttt{for} recorre la cadena. Cada vez que entramos al bucle,
si el caracter \texttt{c} no está en el diccionario, creamos un nuevo
elemento con la clave \texttt{c} y el valor inicial 1 (debido a que
hemos visto esta letra solo una vez). Si \texttt{c} ya está previamente
en el diccionario incrementamos \texttt{d{[}c{]}}.

\index{histograma}

Aquí está la salida del programa:

\begin{verbatim}
{'b': 1, 'r': 2, 'o': 3, 'n': 1, 't': 1, 's': 1, 'a': 1, 'u': 1, 'i': 1}
\end{verbatim}

El histograma indica que las letras ``a'' y ``b'' aparecen solo una vez;
``o'' aparece dos, y así sucesivamente.

\index{método get} \index{get!método}

Los diccionarios tienen un método llamado \texttt{get} que toma una
clave y un valor por defecto. Si la clave aparece en el diccionario,
\texttt{get} regresa el valor correspondiente; si no, regresa el valor
por defecto. Por ejemplo:


\begin{Verbatim}[frame=single]
>>> cuentas = { 'chuck' : 1 , 'annie' : 42, 'jan': 100}
>>> print(cuentas.get('jan', 0))
  100
>>> print(cuentas.get('tim', 0))
  0
\end{Verbatim}


Podemos usar \texttt{get} para escribir nuestro bucle de histograma más
conciso. Puesto que el método \texttt{get} automáticamente maneja el
caso en que una clave no está en el diccionario, podemos reducir cuatro
líneas a una y eliminar la sentencia \texttt{if}.


\begin{python}[frame=single]
palabra = 'brontosaurio'
d = dict()
for c in palabra:
    d[c] = d.get(c,0) + 1
print(d)
\end{python}


%El uso del método \texttt{get} para simplificar este bucle contador termina siendo un ``idioma'' muy utilizado en Python y vamos a utilizarlo muchas veces en el resto del libro. Así que deberías tomar un momento para comparar el bucle utilizando la sentencia \texttt{if} y el operador \texttt{in} con el bucle utilizando el método \texttt{get}. Ambos hacen exactamente lo mismo, pero uno es más breve.

\hypertarget{diccionarios-y-archivos}{%
\section{Diccionarios y archivos}\label{diccionarios-y-archivos}}

Uno de los usos más comunes de un diccionario es contar las ocurrencias
de palabras en un archivo con algún texto escrito. Vamos comenzando con
un archivo de palabras muy simple tomado del texto de \emph{Romeo y
Julieta}.

Para el primer conjunto de ejemplos, vamos a usar una versión más corta
y más simplificada del texto sin signos de puntuación. Después
trabajaremos con el texto de la escena con signos de puntuación
incluidos.

\begin{verbatim}
But soft what light through yonder window breaks
It is the east and Juliet is the sun
Arise fair sun and kill the envious moon
Who is already sick and pale with grief
\end{verbatim}

Vamos a escribir un programa de Python para leer a través de las líneas
del archivo, dividiendo cada línea en una lista de palabras, y después
iterando a través de cada una de las palabras en la línea y contando
cada palabra utilizando un diccionario.

\index{bucles anidados} \index{bucles!anidados}

Verás que tenemos dos bucles \texttt{for}. El bucle externo está leyendo
las líneas del archivo y el bucle interno está iterando a través de cada
una de las palabras en esa línea en particular. Este es un ejemplo de un
patrón llamado \emph{bucles anidados} porque uno de los bucles es el
bucle \emph{externo} y el otro bucle es el bucle \emph{interno}.

Como el bucle interno ejecuta todas sus iteraciones cada vez que el
bucle externo hace una sola iteración, consideramos que el bucle interno
itera ``más rápido'' y el bucle externo itera más lento.

\index{Romeo y Julieta}

La combinación de los dos bucles anidados asegura que contemos cada
palabra en cada línea del archivo de entrada.

%\VerbatimInput{../code3/count1.py}
\begin{python}[frame=single]
fname = input('Ingresa el nombre de archivo: ')
try:
    fhand = open(fname)
except:
    print('El archivo no se puede abrir:', fname)
    exit()

counts = dict()
for line in fhand:
    words = line.split()
    for word in words:
        if word not in counts:
            counts[word] = 1
        else:
            counts[word] += 1

print(counts)
\end{python}

En nuestra sentencia \texttt{else}, utilizamos la alternativa más
compacta para incrementar una variable. \texttt{counts{[}word{]}\ +=\ 1}
es equivalente a \texttt{counts{[}word{]}\ =\ counts{[}word{]}\ +\ 1}.
Cualquiera de los dos métodos puede usarse para cambiar el valor de una
variable en cualquier cantidad. Existen alternativas similares para
\texttt{-=}, \texttt{*=}, y \texttt{/=}.

Cuando ejecutamos el programa, vemos una salida sin procesar que
contiene todos los contadores sin ordenar. El archivo \emph{romeo.txt}
está disponible en PoliformaT.

\begin{Verbatim}[frame=single]
>>> %Run
  Ingresa el nombre de archivo: romeo.txt
  {'and': 3, 'envious': 1, 'already': 1, 'fair': 1,
  'is': 3, 'through': 1, 'pale': 1, 'yonder': 1,
  'what': 1, 'sun': 2, 'Who': 1, 'But': 1, 'moon': 1,
  'window': 1, 'sick': 1, 'east': 1, 'breaks': 1,
  'grief': 1, 'with': 1, 'light': 1, 'It': 1, 'Arise': 1,
  'kill': 1, 'the': 3, 'soft': 1, 'Juliet': 1}
\end{Verbatim}

Es un poco inconveniente ver a través del diccionario para encontrar las
palabras más comunes y sus contadores, así que necesitamos agregar un
poco más de código para mostrar una salida que nos sirva más.

\hypertarget{bucles-y-diccionarios}{%
\section{Bucles y diccionarios}\label{bucles-y-diccionarios}}

\index{bucles y!diccionarios} \index{bucles!y diccionarios}
\index{recorrido}

Si utilizas un diccionario como una secuencia para un bucle
\texttt{for}, esta recorre las claves del diccionario. Este bucle
imprime cada clave y su valor correspondiente:


\begin{python}[frame=single]
contadores = { 'chuck' : 1 , 'annie' : 42, 'jan': 100}
for clave in contadores:
    print(clave, contadores[clave])
\end{python}


Aquí está lo que muestra de salida:

\begin{Verbatim}[frame=single]
jan 100
chuck 1
annie 42
\end{Verbatim}

De nuevo, las claves no están en ningún orden en particular.

\index{idioma}

Podemos utilizar un bucle \texttt{for} de esta manera con diccionarios para implementar diferentes cosas. Por ejemplo, si queremos encontrar todas
las entradas en un diccionario con valor mayor a diez, podemos escribir
el siguiente código:


\begin{python}[frame=single]
contadores = { 'chuck' : 1 , 'annie' : 42, 'jan': 100}
for clave in contadores:
    if contadores[clave] > 10 :
        print(clave, contadores[clave])
\end{python}


El bucle \texttt{for} itera a través de las \emph{claves} del
diccionario, así que debemos utilizar el operador índice para obtener el
\emph{valor} correspondiente a cada clave. Aquí está la salida del
programa:

\begin{Verbatim}[frame=single]
jan 100
annie 42
\end{Verbatim}

Vemos solamente las entradas que tienen un valor mayor a 10.

\index{método de claves} \index{claves!método de}

Si quieres imprimir las claves en orden alfabético, primero haces una
lista de las claves en el diccionario utilizando el método \texttt{keys}
disponible en los objetos de diccionario, y después ordenar esa lista e
iterar a través de la lista ordenada, buscando cada clave e imprimiendo
pares clave-valor ordenados, tal como se muestra a continuación:


\begin{python}[frame=single]
contadores = { 'chuck' : 1 , 'annie' : 42, 'jan': 100}
lst = list(contadores.keys())
print(lst)
lst.sort()
for clave in lst:
    print(clave, contadores[clave])
\end{python}


Así se muestra la salida:

\begin{verbatim}
['jan', 'chuck', 'annie']
annie 42
chuck 1
jan 100
\end{verbatim}

Primero se ve la lista de claves sin ordenar como la obtuvimos del
método \texttt{keys}. Después vemos los pares clave-valor en orden desde
el bucle \texttt{for}.

\hypertarget{anuxe1lisis-avanzado-de-texto}{%
\section{Análisis avanzado de
texto}\label{anuxe1lisis-avanzado-de-texto}}

\index{Romeo y Julieta}

En el ejemplo anterior utilizando el archivo \emph{romeo.txt}, hicimos
el archivo tan simple como fue posible removiendo los signos de
puntuación a mano. El text real tiene muchos signos de puntuación, como:

\begin{verbatim}
But, soft! what light through yonder window breaks?
It is the east, and Juliet is the sun.
Arise, fair sun, and kill the envious moon,
Who is already sick and pale with grief,
\end{verbatim}

Puesto que la función \texttt{split} en Python busca espacios y trata
las palabras como piezas separadas por esos espacios, trataríamos a las
palabras ``soft!'' y ``soft'' como \emph{diferentes} palabras y
crearíamos una entrada independiente para cada palabra en el
diccionario.
Además, como el archivo tiene letras mayúsculas, trataríamos ``who'' y
``Who'' como diferentes palabras con diferentes contadores.

Podemos resolver ambos problemas utilizando los siguientes métodos de cadenas:

1. \texttt{lower}: reemplaza todos los mayúsculas con su minusculas correspondientes:

\begin{Verbatim}[frame=single]
>>> "HeLlOo!".lower()
'helloo!'
\end{Verbatim}

2. \texttt{punctuation}: devuelve los
caracteres que Python considera como ``signos de puntuación'':


\begin{Verbatim}[frame=single]
>>> import string
>>> string.punctuation
'!"#$%&\'()*+,-./:;<=>?@[\\]^_`{|}~'
\end{Verbatim}

3. \texttt{translate}, el más sutil de los métodos. Aquí esta la
documentación para \texttt{translate}:

\texttt{line.translate(str.maketrans(fromstr,\ tostr,\ deletestr))}

\emph{Reemplaza los caracteres en \texttt{fromstr} con el caracter en la
misma posición en \texttt{tostr} y elimina todos los caracteres que
están en \texttt{deletestr}. Los parámetros \texttt{fromstr} y
\texttt{tostr} pueden ser cadenas vacías y el parámetro
\texttt{deletestr} es opcional.}

No vamos a especificar el valor de \texttt{tostr} pero vamos a utilizar
el parámetro \texttt{deletestr} para eliminar todos los signos de
puntuación. 

Hacemos las siguientes modificaciones a nuestro programa:

%\VerbatimInput{../code3/count2.py}
\begin{python}[frame=single]
import string

fname = input('Ingresa el nombre de archivo: ')
try:
    fhand = open(fname)
except:
    print('El archivo no se puede abrir:', fname)
    exit()

counts = dict()
for line in fhand:
    line = line.rstrip()
    line = line.translate(line.maketrans('', '', string.punctuation))
    line = line.lower()
    words = line.split()
    for word in words:
        if word not in counts:
            counts[word] = 1
        else:
            counts[word] += 1

print(counts)
\end{python}

Parte de aprender el ``Arte de Python'' o ``Pensamiento Pythónico'' es
entender que Python muchas veces tiene funciones internas para muchos
problemas de análisis de datos comunes. A través del tiempo, verás
suficientes códigos de ejemplo y leerás lo suficiente en la
documentación para saber dónde buscar si alguien escribió algo que haga
tu trabajo más fácil.

Lo siguiente es una versión reducida de la salida:

\begin{verbatim}
Ingresa el nombre de archivo: romeo-full.txt
{'swearst': 1, 'all': 6, 'afeard': 1, 'leave': 2, 'these': 2,
'kinsmen': 2, 'what': 11, 'thinkst': 1, 'love': 24, 'cloak': 1,
a': 24, 'orchard': 2, 'light': 5, 'lovers': 2, 'romeo': 40,
'maiden': 1, 'whiteupturned': 1, 'juliet': 32, 'gentleman': 1,
'it': 22, 'leans': 1, 'canst': 1, 'having': 1, ...}
\end{verbatim}

Interpretar los datos a través de esta salida es aún difícil, y podemos
utilizar Python para darnos exactamente lo que estamos buscando, pero
para que sea así, necesitamos aprender acerca de las \emph{tuplas} en
Python. Vamos a retomar este ejemplo una vez que aprendamos sobre
tuplas.

\hypertarget{depuraciuxf3n}{%
\section{Depuración}\label{depuraciuxf3n}}

\index{depuración}

Conforme trabajes con conjuntos de datos más grandes puede ser
complicado depurar imprimiendo y revisando los datos a mano. Aquí hay
algunas sugerencias para depurar grandes conjuntos de datos:

\begin{description}
\item[Reducir la entrada]
Si es posible, trata de reducir el tamaño del conjunto de datos. Por
ejemplo, si el programa lee un archivo de texto, comienza solamente con
las primeras 10 líneas, o con el ejemplo más pequeño que puedas
encontrar. Puedes ya sea editar los archivos directamente, o (mejor)
modificar el programa para que solamente lea las primeras \texttt{n}
número de líneas.

Si hay un error, puedes reducir \texttt{n} al valor más pequeño que
produce el error, y después incrementarlo gradualmente conforme vayas
encontrando y corrigiendo errores.
\item[Revisar extractos y tipos]
En lugar de imprimir y revisar el conjunto de datos completo, considera
imprimir extractos de los datos: por ejemplo, el número de elementos en
un diccionario o el total de una lista de números.

Una causa común de errores en tiempo de ejecución es un valor que no es
el tipo correcto. Para depurar este tipo de error, generalmente es
suficiente con imprimir el tipo de un valor.
\item[Escribe auto-verificaciones]
Algunas veces puedes escribir código para revisar errores
automáticamente. Por ejemplo, si estás calculando el promedio de una
lista de números, podrías verificar que el resultado no sea más grande
que el elemento más grande de la lista o que sea menor que el elemento
más pequeño de la lista. Esto es llamado ``prueba de sanidad'' porque
detecta resultados que son ``completamente ilógicos''.
\index{prueba sanidad} \index{prueba consistencia}

Otro tipo de prueba compara los resultados de dos diferentes cálculos
para ver si son consistentes. Esto es conocido como ``prueba de
consistencia''.
\item[Imprimir una salida ordenada]
Dar un formato a los mensajes de depuración puede facilitar encontrar un
error.
\end{description}

De nuevo, el tiempo que inviertas haciendo una buena estructura puede
reducir el tiempo que inviertas en depurar. \index{estructurar}

\hypertarget{glosario}{%
\section{Glosario}\label{glosario}}

\begin{description}
\tightlist
\item[bucles anidados]
Cuando hay uno o más bucles ``dentro'' de otro bucle. Los bucles
internos terminan de ejecutar cada vez que el bucle externo ejecuta una
vez. \index{bucles anidados} \index{anidados!bucles}
\item[búsqueda]
Una operación de diccionario que toma una clave y encuentra su valor
correspondiente. \index{búsqueda}
\item[clave]
Un objeto que aparece en un diccionario como la primera parte de un par
clave-valor. \index{clave}
\item[diccionario]
Una asociación de un conjunto de claves a sus valores correspondientes.
\index{diccionario}
\item[elemento]
Otro nombre para un par clave-valor. \index{elemento!diccionario}
\item[función hash]
Una función utilizada por una tabla hash para calcular la localización
de una clave. \index{función hash}
\item[histograma]
Un set de contadores. \index{histograma}
\item[implementación]
Una forma de llevar a cabo un cálculo. \index{implementación}
\item[par clave-valor]
La representación de una asociación de una clave a un valor.
\index{par clave-valor}
\item[tabla hash]
El algoritmo utilizado para implementar diccionarios en Python.
\index{tabla hash}
\item[valor]
Un objeto que aparece en un diccionario como la segunda parte de un par
clave-valor. Esta definición es más específica que nuestro uso previo de
la palabra ``valor''. \index{valor}
\end{description}



