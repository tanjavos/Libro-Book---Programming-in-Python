
\section{Ejercicios}


\begin{exercise}
El alfabeto de la {\em International Civil Aviation Organization }(ICAO) asigna un código de palabras a las letras del alfabeto inglés (alfa para la A, bravo para la B, etc.), de forma que las combinaciones de letras críticas se pueden pronunciar y pueden ser entendidas por aquellos que transmiten y reciben mensajes de voz por radio o teléfono, independientemente de su idioma nativo. 
Imagina que tenemos el siguiente diccionario en Python asignado en la variable \texttt{dic\_ICAO}:

\begin{small}
\begin{python}
dic_ICAO = {
    'a':'alfa',
    'b':'bravo',
    'c':'charlie',
    'd':'delta',
    'e':'echo',
    'f':'foxtrot',
    'g':'golf',
    'h':'hotel',
    'i':'india',
    'j':'juliett',
    'k':'kilo',
    'l':'lima',
    'm':'mike',
    'n':'november',
    'o':'oscar',
    'p':'papa',
    'q':'quebec',
    'r':'romeo',
    's':'sierra',
    't':'tango',
    'u':'uniform',
    'v':'victor',
    'w':'whiskey',
    'x':'x-ray',
    'y':'yankee',
    'z':'zulu'
    }
\end{python}
\end{small}

Diseña una función que reciba un texto y sea capaz de traducirlo en la lista de palabras ICAO. Puedes usar la siguiente plantilla (template) que tambien esta en PoliformaT.

\begin{python}
def traducir_a_ICAO (texto):

    """
    Esta función traduce un texto al alfabeto ICAO definido en 
    el diccionario dic_ICAO.

    Entradas: Un texto a traducir
    Salidas:  Un texto traducido a ICAO
    Restricciones: solo funciona con texto de tipo str.
    """

#    <TU SOLUCIÓN AQUI>


@pytest.mark.parametrize("testcase, entrada, salida_esperada",[

#    <TUS CASOS DE TEST AQUI>
    
])

def test_traducir_a_ICAO(testcase, entrada, salida_esperada):
    assert traducir_a_ICAO(entrada) == salida_esperada,\
           "caso {0}".format(testcase)


\end{python}
\end{exercise}


\begin{exercise}
Considera la siguiente función en Python:

\begin{python}
def traducir(dic, source):
    result = dic[source[0]]
    for i in range(1, len(source)):
        result = result + dic[source[i]]
    return result
\end{python}

Imagina también que tenemos los siguientes diccionarios definidos:

\begin{small}
\begin{python}
dic_ej1 = {
    1: 'one',
    2: 'two',
    3: 'three'
    }

dic_ej2 = {
    1: [1],
    2: [2],
    3: [3]
    }

dic_ej3 = {
    1: 1,
    2: 2,
    3: 3
    }

dic_ej4 = {
    '1': 1,
    '2': 2,
    '3': 3
    }
\end{python}
\end{small}

¿Qué respuesta da Python para las siguientes instrucciones?

\begin{small}
\begin{Verbatim}[frame=single]
1. >>> traducir (dic_ej1, (1,2,1))
2. >>> traducir (dic_ej1, [1,2,1])
3. >>> traducir (dic_ej2, [1,2,1])
4. >>> traducir (dic_ej2, 123)
5. >>> traducir (dic_ej3, [1,2,1])
6. >>> traducir (dic_ej4, [1,2,1])
7. >>> traducir (dic_ej4, "123")
8. >>> traducir (dic_ej4, ['1', '2', '3'])
\end{Verbatim}
\end{small}
\end{exercise}

\solucion{
>>> traducir (dic_ej1, (1,2,1))
'onetwoone'
>>> traducir (dic_ej1, [1,2,1])
'onetwoone'
>>> traducir (dic_ej2, [1,2,1])
[1, 2, 1]
>>> traducir (dic_ej2, 123)
result = dic[input[0]]
TypeError: 'int' object is not subscriptable
>>> traducir (dic_ej3, [1,2,1])
4
>>> traducir (dic_ej4, [1,2,1])
result = dic[input[0]]
KeyError: 1
>>> traducir (dic_ej4, "123")
6
>>> traducir (dic_ej4, ['1', '2', '3'])
6
}


\begin{exercise}
Escribe la documentación para la función \pythoninline{traducir}, que hace la función para que queda claro para que parámetros funciona y para que parámetros lanza excepciones y cuales.
\end{exercise} 


\begin{exercise}
Los siguientes tests son para testear algunos de los ejemplos de más arriba (i.e. 1 hasta 8). 

\begin{small}
\begin{python}
@pytest.mark.parametrize("testcase, input1, input2, output",[
(a, dic_ej2, [1,2,1], [1,2,1]),
(b, dic_ej1, (1,2,1), 'onetwoone'),   
(c, dic_ej1, [1,2,1], 'onetwoone'),  
(d, dic_ej3, [1,2,1], 4),
(e, dic_ej4, ['1', '2', '3'], 6)
(f, dic_ej4, "123", 6),
])

def test_traducir(testcase, input1, input2, output):
    assert traducir(input1, input2) == output,\
           "caso {0}".format(testcase)
\end{python}
\end{small}     

¿Con qué números correspondan \pythoninline{a, b, c, d, e, f}?
\end{exercise} 


\solucion{b=1, c=2, a=3, d=5, f= 7, e = 8}


\begin{exercise}
Hemos visto que el metodo \pythoninline{traducir} lanza un \pythoninline{TypeError} cuando el \pythoninline{source} tiene un tipo que no es {\em subscriptable}. Tambien vimos un
\pythoninline{KeyError} cuando el clave con que intentamos acceder al diccionario no coincide con el tipo del diccionario.

Para escribir \pythoninline{pytest} que verifican que salen estas excepciones se puede escribir los siguientes tests:

\begin{small}
\begin{python}
@pytest.mark.parametrize("testcase, input1, input2, output",[
(g, dic_ej2, 123, pytest.raises(TypeError)),   
(h, dic_ej4, [1,2,1], pytest.raises(KeyError))
])
   
def test_traducir_exceptions(testcase, input1, input2, output):
    with output:
        traducir(input1, input2)
\end{python}
\end{small}     
¿Con qué números correspondan \pythoninline{g, h}?
\end{exercise} 

\begin{exercise}
Vamos a implementar un programa para controlar el stock de una frutería. Para ello, tenemos dos diccionarios: 

\begin{small}
\begin{python}
precio = {
    "platano": 3,
    "manzana": 2,
    "naranja": 2.5,
    "pera": 3.5
    }

stock = {
    "platano": 20,
    "manzana": 30,
    "naranja": 15,
    "pera": 30
    }
\end{python}
\end{small}

Empezamos con una función \pythoninline{valor_del_stock} que calcula, como indica su nombre, el valor que tiene el stock.

¿Con qué valores vas a ejecutar tu función para testearlo bien?
\end{exercise} 

\begin{exercise}  Ahora implementa una función \pythoninline{venta} que simula una venta.
Para ello, la función recibe como parámetros el nombre de la fruta que se vende y la cantidad. La función deberá descontar la cantidad vendida del stock. Si no hay stock, se mostrará un mensaje de error. 

¿Con qué valores vas a ejecutar tu función para testearlo bien?
\end{exercise} 

\begin{exercise} Ahora implementa una función \pythoninline{compra} que simula una compra.
Para ello, la función recibe como parametros el nombre de la fruta que se ha comprado y la cantidad. La función deberá actualizar la cantidad comprado al stock. 
Si este producto no ha aparecido antes en el stock (i.e. melones), hay que añadirlo.

¿Con qué valores vas a ejecutar tu función para testearlo bien?
\end{exercise} 

\begin{exercise}
Una anagrama es una palabra o frase que se forma al reorganizar las letras de otra palabra o frase. Por ejemplo, ''amor'' es un anagrama de ''roma'', y ''frase'' 
es una anagrama de ''fresa''. Escriba una función \pythoninline{is_anagram (word1, word2)} que verifique si las dos palabras son anagramas entre sí. Si es así, la función debería devolver True, y False de lo contrario. La función no debe distinguir entre letras mayúsculas y minúsculas.
Puedes usar las siguientes ejemplos para testear la función con pytest:

\begin{longtable}{|l|l|l|}
\hline
testcase número & input & output esperado   \\ \hline
1  & (frase, fresa)   & True \\ 
2  & (panel, nepal)     & True\\
3 & (Alumno, Molan)     & False \\
4 & (Hola, Caracola)      & False \\
5 & (Rectificable, certificable)  & True \\
\hline
\end{longtable}

\end{exercise} 

\begin{exercise}
Imagine un pequeño hotel con cuatro habitaciones de cuatro camas (con los números elegidos arbitrariamente 101, 102, 201 y 202). Vamos a escribir un programa Python para el personal del hotel para ayudarlos a realizar un seguimiento de la ocupación de la habitación y registrar la entrada y salida de los huéspedes. 

El código Python para la interacción del usuario ya existe:

\begin{small}
\begin{python}
room_occupancy = {101:[], 102:[], 201:[], 202:[]} 
while True:
    print("These are your options:")
    print("1 - View current room occupancy.")
    print("2 - Check guest in.")
    print("3 - Check guest out.")
    print("4 - Exit program.")
    choice = input("Please choose what you want to do: ") 
    if choice == "1":
        print_occupancy(room_occupancy)
    elif choice == "2":
        guest = input("Enter name of guest: ")
        room = int(input("Enter room number: "))
        check_in(room_occupancy, guest, room)
    elif choice == "3":
        guest = input("Enter name of guest: ")
        room = int(input("Enter room number: "))
        check_out(room_occupancy, guest, room)
    elif choice == "4":
        print("Goodbye!")
        break
    else:
        print("Invalid input, try again.")
\end{python}
\end{small}

Vamos a escribir un par de funciones más:


\begin{itemize}
\item \pythoninline{print_occupancy} 
simplemente debe imprimir una lista de todas las habitaciones y los huéspedes que se registran actualmente.

\item \pythoninline{check_in} 
debería agregar un invitado a una habitación. Si se proporciona un número de habitación no existente o si la habitación elegida ya está llena, se debe imprimir el mensaje correspondiente. Puede haber dos invitados con el mismo nombre en una habitación

\item \pythoninline{check_out} 
debe eliminar a un invitado de una habitación. Si se pasa un número de habitación o nombre de invitado incorrecto, se debe imprimir el mensaje correspondiente
\end{itemize}

\end{exercise} 