
\hypertarget{las-tuplas-son-inmutables}{%
\section{Las tuplas son inmutables}\label{las-tuplas-son-inmutables}}

\index{tupla} \index{tipo!tupla} \index{secuencia}

Una tupla\footnote{Dato curioso: La palabra ``tuple'' proviene de los
  nombres dados a secuencias de números de distintas longitudes: simple,
  doble, triple, cuádruple, quíntuple, séxtuple, séptuple, etc.} es una
secuencia de valores similar a una lista. Los valores guardados en una
tupla pueden ser de cualquier tipo, y son indexados por números enteros.
La principal diferencia es que las tuplas son \emph{inmutables}. Las
tuplas además son \emph{comparables}
de modo que las listas de tuplas se pueden ordenar y también usar tuplas
como valores para las claves en diccionarios de Python.

Sintácticamente, una tupla es una lista de valores separados por comas:

\begin{Verbatim}[frame=single]
>>> t = 'a', 'b', 'c', 'd', 'e'
\end{Verbatim}

Aunque no es necesario, es común encerrar las tuplas entre paréntesis para ayudarnos a identificarlas rápidamente cuando revisemos código de Python:

\index{paréntesis!tuplas}


\begin{Verbatim}[frame=single]
>>> t = ('a', 'b', 'c', 'd', 'e')
\end{Verbatim}


Para crear una tupla con un solo elemento, es necesario incluir una coma
al final:

\index{único} \index{tupla!único}


\begin{Verbatim}[frame=single]
>>> t1 = ('a',)
>>> type(t1)
<type 'tuple'>
\end{Verbatim}


Sin la coma, Python considera
\texttt{(\textquotesingle{}a\textquotesingle{})} como una expresión con
una cadena entre paréntesis que es evaluada como de tipo cadena
(string):


\begin{Verbatim}[frame=single]
>>> t2 = ('a')
>>> type(t2)
<type 'str'>
\end{Verbatim}


Otra forma de construir una tupla es utilizando la función interna \texttt{tuple}. Sin argumentos, ésta crea una tupla vacía:

\index{función tupla} \index{tupla!función}


\begin{Verbatim}[frame=single]
>>> t = tuple()
>>> print(t)
()
\end{Verbatim}


Si el argumento es una secuencia (cadena, lista, o tupla), el resultado
de la llamada a \texttt{tuple} es una tupla con los elementos de la
secuencia:


\begin{Verbatim}[frame=single]
>>> t = tuple('altramuces')
>>> print(t)
('a', 'l', 't', 'r', 'a', 'm', 'u', 'c', 'e', 's')
\end{Verbatim}


Dado que \texttt{tuple} es el nombre de un constructor, debería evitarse
su uso como nombre de variable.

La mayoría de los operadores de listas también funcionan en tuplas. El
operador corchete para la indexación de sus elementos:

\begin{Verbatim}[frame=single]
>>> t = ('a', 'b', 'c', 'd', 'e')
>>> print(t[0])
'a'
\end{Verbatim}

Y el operador de concatenación (\pythoninline{+}) funciona con tuplas:

\begin{Verbatim}[frame=single]
>>> (1,2) + (3,4,5)
  (1, 2, 3, 4, 5)
>>> (2,) + (9,)
  (2, 9)
\end{Verbatim}

Y el operador de repetición (\pythoninline{*}) funciona con tuplas:

\begin{Verbatim}[frame=single]
>>> 3 * (1,)
  (1, 1, 1)
>>> 5 * (1,2,3)
  (1, 2, 3, 1, 2, 3, 1, 2, 3, 1, 2, 3, 1, 2, 3)
\end{Verbatim}


Y el operador de calcular el length (\pythoninline{len}) funciona con tuplas:

\begin{Verbatim}[frame=single]
>>> len ((4,5,5,6,7,8))
  6
>>> len()
  TypeError: len() takes exactly one argument (0 given)
>>> len(()) 
  0
\end{Verbatim}


Y el operador de rebanado (slice) selecciona un rango de elementos.

\index{operador rebanado} \index{rebanado!operador}
\index{rebanado!tupla} \index{tupla!rebanado}


\begin{Verbatim}[frame=single]
>>> print(t[1:3])
('b', 'c')
\end{Verbatim}


Pero si se intenta modificar uno de los elementos de la tupla, se
produce un error:

\index{excepción!TypeError} \index{TypeError}
\index{asignación de objeto} \index{objeto!asignación de}


\begin{Verbatim}[frame=single]
>>> t[0] = 'A'
TypeError: object doesn't support item assignment
\end{Verbatim}


No se puede modificar los elementos de una tupla, pero sí se puede
reemplazar una tupla por otra:


\begin{Verbatim}[frame=single]
>>> t = ('A',) + t[1:]
>>> print(t)
('A', 'b', 'c', 'd', 'e')
\end{Verbatim}


\hypertarget{comparaciuxf3n-de-tuplas}{%
\section{Comparación de tuplas}\label{comparaciuxf3n-de-tuplas}}

\index{tupla!comparación} \index{comparación!tupla}
\index{método de ordenamiento} \index{de ordenamiento!método}

Los operadores de comparación funcionan con tuplas y otras secuencias.
Python comienza comparando el primer elemento de cada secuencia. Si
ambos elementos son iguales, pasa al siguiente elemento y así
sucesivamente, hasta que encuentra elementos diferentes. Los elementos
subsecuentes no son considerados (aunque sean muy grandes).


\begin{Verbatim}[frame=single]
>>> (0, 1, 2) < (0, 3, 4)
True
>>> (0, 1, 2000000) < (0, 3, 4)
True
\end{Verbatim}

La función \texttt{sort} funciona de la misma manera. Ordena inicialmente por el primer elemento, pero en el caso de que ambos
elementos sean iguales, ordena por el segundo elemento, y así sucesivamente.


\section{Patrón de diseño: DSU}\label{DSU}

La comparación de tuplas da lugar a un patrón de diseño llamado \emph{DSU} que podemos usar para ordenar secuencias de elementos según una determinada propiedad de estos elementos que usamos como índice. DSU consiste en:

\begin{description}
\item[Decorate] (Decora)
una secuencia, construyendo una lista de tuplas con uno o más índices
ordenados precediendo los elementos de la secuencia,
\item[Sort] (Ordena)
la lista de tuplas utilizando la función interna \texttt{sort}, y
\item[Undecorate] (Quita la decoración)
extrayendo los elementos ordenados de la secuencia.
\end{description}

\index{patrón DSU} \index{DSU!patrón}
\index{patrón decorate-sort-undecorate}
\index{decorate-sort-undecorate!patrón} \index{Romeo y Julieta}

Por ejemplo, suponiendo una lista de palabras que se quieren ordenar de
la más larga a la más corta:

\begin{python}
txt = 'Pero qué luz se deja ver allí'
palabras = txt.split()

#Decorar las palabras de la lista con su len
t = list()
for palabra in palabras:
    t.append((len(palabra), palabra))

#Ordenar
t.sort(reverse=True)

#Undecorar, quitando el len
res = list()
for longitud, palabra in t:
    res.append(palabra)

print(res)
\end{python}

El primer bucle genera una lista de tuplas, donde cada tupla es una
palabra precedida por su longitud.

\texttt{sort} compara el primer elemento (longitud) primero, y solamente
considera el segundo elemento para desempatar. El argumento clave
\texttt{reverse=True} indica a \texttt{sort} que debe ir en orden
decreciente.

\index{argumento clave} \index{clave!argumento} \index{recorrido}

El segundo bucle recorre la lista de tuplas y construye una lista de
palabras en orden descendente según la longitud. Las palabras de cuatro
letras están ordenadas en orden alfabético \emph{inverso}, así que
``deja'' aparece antes que ``allí'' en la siguiente lista.

La salida del programa es la siguiente:

\begin{Verbatim}
['deja', 'allí', 'Pero', 'ver', 'qué', 'luz', 'se']
\end{Verbatim}

Por supuesto, la línea pierde mucho de su impacto poético cuando se
convierte en una lista de Python y se almacena en orden descendente
según la longitud de las palabras.


\section{Asignación de tuplas}

A menudo es útil intercambiar los valores de dos variables.
Con asignaciones convencionales, tienes que utilizar una variable
temporal.  Por ejemplo, para intercambiar \texttt{a} y \texttt{b}:

\begin{Verbatim}
>>> temp = a
>>> a = b
>>> b = temp
\end{Verbatim}
%
Esta solución es incómoda; la \textbf{asignación de tupla} es más elegante:

\begin{Verbatim}
>>> a, b = b, a
\end{Verbatim}
%
El lado izquierdo es una tupla de variables; el lado derecho es una tupla de
expresiones.  Cada valor es asignado a su respectiva variable.
Todas las expresiones en el lado derecho son evaluadas antes que cualquiera
de las asignaciones.

Ambos lados de la sentencia son tuplas, pero el lado izquierdo es una
tupla de variables; el lado derecho es una tupla de expresiones. Cada
valor en el lado derecho es asignado a su respectiva variable en el lado
izquierdo. Todas las expresiones en el lado derecho son evaluadas antes
de realizar cualquier asignación.

El número de variables en el lado izquierdo y el número de valores en el
lado derecho deben ser iguales:

\index{excepción!ValueError} \index{ValueError}

\begin{Verbatim}[frame=single]
>>> a, b = 1, 2, 3
  ValueError: too many values to unpack
\end{Verbatim}

Generalizando más, el lado derecho puede ser cualquier tipo de secuencia
(cadena, lista, o tupla). De esta manera tenemos un mecanismo de empaquetado/desempaquetado utilizando tuplas.

Por ejemplo, para dividir una dirección de
e-mail en nombre de usuario y dominio, se podría escribir:

\index{método split} \index{split!método} \index{dirección e-mail}

\begin{Verbatim}[frame=single]
>>> dir = 'monty@python.org'
>>> nombreus, dominio = dir.split('@')
\end{Verbatim}


El valor de retorno de \texttt{split} es una lista con dos elementos; el
primer elemento es asignado a \texttt{nombreus}, el segundo a
\texttt{dominio}.


\begin{Verbatim}[frame=single]
>>> print(nombreus)
  monty
>>> print(dominio)
  python.org
\end{Verbatim}


En el siguiente ejemplo tenemos una lista de dos elementos (la cual es una
secuencia) y asignamos el primer y segundo elementos de la secuencia a
las variables \texttt{x} y \texttt{y} en una única sentencia.

\begin{Verbatim}[frame=single]
>>> m = [ 'pásalo', 'bien' ]
>>> x, y = m
>>> x
  'pásalo'
>>> y
  'bien'
>>>
\end{Verbatim}

No es magia, Python traduce \emph{aproximadamente} la sintaxis de
asignación de la tupla de este modo::\footnote{Python no traduce la
sintaxis literalmente. Por ejemplo, si se trata de hacer esto con un
diccionario, no va a funcionar como se podría esperar.}


\begin{Verbatim}[frame=single]
>>> m = [ 'pásalo', 'bien' ]
>>> x = m[0]
>>> y = m[1]
>>> x
  'pásalo'
>>> y
  'bien'
>>>
\end{Verbatim}

Estilísticamente, cuando se utiliza una tupla en el lado izquierdo de la
asignación, se omiten los paréntesis, pero lo que se muestra a
continuación es una sintaxis igualmente válida:


\begin{Verbatim}[frame=single]
>>> m = [ 'pásalo', 'bien' ]
>>> (x, y) = m
>>> x
'pásalo'
>>> y
'bien'
>>>
\end{Verbatim}


\section{Tuplas como valores de retorno}
\index{tupla}
\index{valor!tupla}
\index{valor de retorno!tupla}
\index{función, tupla como valor de retorno}

Estrictamente hablando, una función puede devolver solo un valor de retorno, pero
si el valor es una tupla, el efecto es el mismo que devolver
múltiples valores.  Por ejemplo, si quieres dividir dos enteros
y calcular el cociente y el resto, es ineficiente
calcular \texttt{x//y} y luego \texttt{x\%y}.  Es mejor calcular
ambos al mismo tiempo.
\index{divmod}

La función incorporada \texttt{divmod} toma dos argumentos y
devuelve una tupla de dos valores: el cociente y el resto.
Puedes almacenar el resultado como una tupla:

\begin{Verbatim}[frame=single]
>>> t = divmod(7, 3)
>>> t
(2, 1)
\end{Verbatim}
%
O bien, utilizar asignación de tupla para almacenar los elementos por separado:
\index{asignación!de tupla}
\index{tupla!asignación de}

\begin{Verbatim}[frame=single]
>>> cociente, resto = divmod(7, 3)
>>> cociente
2
>>> resto
1
\end{Verbatim}
%
Aquí hay un ejemplo de una función que devuelve una tupla:

\begin{Verbatim}[frame=single]
def min_max(t):
    return min(t), max(t)
\end{Verbatim}
%
\texttt{max} y \texttt{min} son funciones incorporadas que encuentran
los elementos más grande y más pequeño en una secuencia.  \verb"min_max"
calcula ambos y devuelve una tupla de dos valores.
\index{max, función}
\index{función!max}
\index{min, función}
\index{función!min}

\section{Funciones con una cantidad variable de parámetros}
\label{gather}

Las funciones pueden tomar una cantidad variable de argumentos. Un nombre de parámetro
que comienza con \texttt{*} hace una \textbf{reunión} de parametros en
una tupla.  Por ejemplo, \texttt{printall}
toma cualquier cantidad de argumentos y los imprime:

\begin{python}[frame=single]
def printall(*args):
    print(args)
\end{python}
%
El parámetro de reunión puede tener cualquier nombre que quieras, pero \texttt{args} es
convencional.  Aquí se muestra cómo opera la función:

\begin{Verbatim}[frame=single]
>>> printall(1, 2.0, '3')
  (1, 2.0, '3')
>>> printall(2.3, '3', 'abc', 34)
  (2.3, '3', 'abc', 34)
\end{Verbatim}
%
El complemento de la reunión es la \textbf{dispersión}.  Si tienes una
secuencia de valores y quieres pasarlo a una función
como múltiples argumentos, puedes utilizar el operador \texttt{*}.
Por ejemplo, \texttt{divmod} toma exactamente dos argumentos;
no funciona con una tupla:
\index{dispersión}
\index{argumento!dispersión de}
\index{TypeError}
\index{excepción!TypeError}

\begin{Verbatim}[frame=single]
>>> t = (7, 3)
>>> divmod(t)
TypeError: divmod expected 2 arguments, got 1
\end{Verbatim}
%
Sin embargo, si dispersas la tupla, funciona:

\begin{Verbatim}[frame=single]
>>> divmod(*t)
(2, 1)
\end{Verbatim}
%
Muchas de las funciones incorporadas utilizan
tuplas de argumentos de longitud variable.  Por ejemplo, \texttt{max}
y \texttt{min} pueden tomar cualquier cantidad de argumentos:
\index{max, función}
\index{función!max}
\index{min, función}
\index{función!min}

\begin{Verbatim}[frame=single]
>>> max(1, 2, 3)
3
\end{Verbatim}
%
Pero \texttt{sum} no puede.
\index{sum, función}
\index{función!sum}

\begin{Verbatim}[frame=single]
>>> sum(1, 2, 3)
TypeError: sum expected at most 2 arguments, got 3
\end{Verbatim}
%
Una función llamada que tome cualquier cantidad de argumentos y devuelva su suma lo podemos escribir nosotros como por ejemplo abajo:

\begin{python}
def suma_todo(*items):
    if (not items):
        return items
    
    resultado = items[0]
    for item in items[1:]:
        resultado = resultado + item
    return resultado
\end{python}


Lo primero que hacemos es verificar si recibimos algún argumento. Si no, volvemos
\pythoninline{items}, una tupla vacía. Esto es necesario porque el resto de la función requiere que
sabemos que tenemos un elemento en el índice 0. Observa que no buscamos una tupla vacía comparándola con () o marcando
que su longitud es 0. Más bien, decimos 
\pythoninline{(not items)}, lo que solicita el valor booleano de nuestra tupla. Debido a que una secuencia de Python vacía es falsa en un contexto booleano, obtenemos
False si \pythoninline{items} está vacío y True en caso contrario.
\newpage

\begin{Verbatim}[frame=single]
>>> suma_todo()
()
>>> suma_todo(2)
2
>>> suma_todo(10, 20, 30, 40)
100
>>> suma_todo('a', 'b', 'c', 'd')
'abcd'
>>> suma_todo([10, 20, 30], [40, 50, 60], [70, 80])
[10, 20, 30, 40, 50, 60, 70, 80]
\end{Verbatim}

\hypertarget{diccionarios-y-tuplas}{%
\section{Diccionarios y tuplas}\label{diccionarios-y-tuplas}}

\index{diccionario} \index{método items} \index{items!método}
\index{par clave-valor}

Los diccionarios tienen un método llamado \texttt{items} que podemos convertir a una
lista de tuplas, donde cada tupla es un par clave-valor:


\begin{Verbatim}[frame=single]
>>> d = {'a':10, 'b':1, 'c':22}
>>> t = list(d.items())
>>> print(t)
[('b', 1), ('a', 10), ('c', 22)]
\end{Verbatim}


Como sería de esperar en un diccionario, los elementos no tienen ningún
orden en particular.

Aun así, puesto que la lista de tuplas es una lista, y las tuplas son
comparables, ahora se puede ordenar la lista de tuplas. Convertir un
diccionario en una lista de tuplas es una forma de obtener el contenido
de un diccionario ordenado según sus claves:

\begin{Verbatim}[frame=single]
>>> d = {'a':10, 'b':1, 'c':22}
>>> t = list(d.items())
>>> t
[('b', 1), ('a', 10), ('c', 22)]
>>> t.sort()
>>> t
[('a', 10), ('b', 1), ('c', 22)]
\end{Verbatim}

La nueva lista está ordenada en orden alfabético ascendente de acuerdo
al valor de sus claves.

La combinación de \texttt{items}, asignación de tuplas, y \texttt{for},
produce un buen patrón de diseño de código para recorrer las claves y
valores de un diccionario en un único bucle:

\begin{Verbatim}[frame=single]
for clave, valor in list(d.items()):
    print(valor, clave)
\end{Verbatim}


Este bucle tiene dos \emph{variables de iteración}, debido a que
\texttt{items} retorna una lista de tuplas y \texttt{clave,\ valor} es
una asignación en tupla que itera sucesivamente a través de cada uno de
los pares clave-valor del diccionario.

Para cada iteración a través del bucle, tanto \texttt{clave} y
\texttt{valor} van pasando al siguiente par clave-valor del diccionario
(todavía en orden de dispersión).

La salida de este bucle es:

\begin{Verbatim}[frame=single]
10 a
1 b
22 c
\end{Verbatim}


De nuevo, las claves están en orden de dispersión (es decir, ningún
orden en particular).

Si se combinan esas dos técnicas, se puede imprimir el contenido de un
diccionario ordenado por el \emph{valor} almacenado en cada par
clave-valor.

Para hacer esto, primero se crea una lista de tuplas donde cada tupla es
\texttt{(valor,\ clave)}. El método \texttt{items} dará una lista de
tuplas \texttt{(clave,\ valor)}, pero esta vez se pretende ordenar por
valor, no por clave. Una vez que se ha construido la lista con las
tuplas clave-valor, es sencillo ordenar la lista en orden inverso e
imprimir la nueva lista ordenada.


\begin{Verbatim}[frame=single]
>>> d = {'a':10, 'b':1, 'c':22}
>>> l = list()
>>> for clave, valor in d.items() :
...     l.append( (valor, clave) )
...
>>> l
[(10, 'a'), (1, 'b'), (22, 'c')]
>>> l.sort(reverse=True)
>>> l
[(22, 'c'), (10, 'a'), (1, 'b')]
>>>
\end{Verbatim}

Al construir cuidadosamente la lista de tuplas para tener el valor como
el primer elemento de cada tupla, es posible ordenar la lista de tuplas
y obtener el contenido de un diccionario ordenado por valor.


\hypertarget{uso-de-tuplas-como-claves-en-diccionarios}{%
\section{Uso de tuplas como claves en
diccionarios}\label{uso-de-tuplas-como-claves-en-diccionarios}}

\index{tupla!como clave en diccionario} \index{hashable}

Dado que las tuplas son \textbf{dispersables} \emph{(hashable)} y las
listas no, si se quiere crear una clave \textbf{compuesta} para usar en
un diccionario, se debe utilizar una tupla como clave.

Usaríamos por ejemplo una clave compuesta si quisiéramos crear un
directorio telefónico que mapea pares appellido, nombre con números
telefónicos. Asumiendo que hemos definido las variables
\texttt{apellido}, \texttt{nombre}, y \texttt{número}, podríamos
escribir una sentencia de asignación de diccionario como sigue:


\begin{Verbatim}[frame=single]
directorio[apellido,nombre] = numero
\end{Verbatim}


La expresión entre corchetes es una tupla. Podríamos utilizar asignación
de tuplas en un bucle \texttt{for} para recorrer este diccionario.

\index{tupla!entre corchetes}

\begin{Verbatim}[frame=single]
for apellido, nombre in directorio:
    print(nombre, apellido, directorio[apellido,nombre])
\end{Verbatim}


Este bucle recorre las claves en \texttt{directorio}, las cuales son
tuplas. Asigna los elementos de cada tupla a \texttt{apellido} y
\texttt{nombre}, después imprime el nombre y el número telefónico
correspondiente.








\hypertarget{depuraciuxf3n}{%
\section{Depuración}\label{depuraciuxf3n}}

\index{depuración} \index{estructura de datos} \index{error de modelado}
\index{de modelado!error}

Las listas, diccionarios y tuplas son conocidas de forma genérica como
\emph{estructuras de datos}; en este capítulo estamos comenzando a ver
estructuras de datos compuestas, como listas de tuplas, y diccionarios
que contienen tuplas como claves y listas como valores. Las estructuras
de datos compuestas son útiles, pero también son propensas a lo que yo
llamo \emph{errores de modelado}; es decir, errores causados cuando una
estructura de datos tiene el tipo, tamaño o composición incorrecto, o
quizás al escribir una parte del código se nos olvidó cómo era el
modelado de los datos y se introdujo un error. Por ejemplo, si estás
esperando una lista con un entero y recibes un entero solamente (no en
una lista), no funcionará.

\hypertarget{glosario}{%
\section{Glosario}\label{glosario}}

\begin{description}
\tightlist
\item[comparable]
Un tipo en el cual un valor puede ser revisado para ver si es mayor que,
menor que, o igual a otro valor del mismo tipo. Los tipos que son
comparables pueden ser puestos en una lista y ordenados.
\index{comparable}
\item[estructura de datos]
Una collección de valores relacionados, normalmente organizados en
listas, diccionarios, tuplas, etc. \index{estructura de datos}
\item[DSU]
Abreviatura de ``decorate-sort-undecorate (decorar-ordenar-quitar la
decoración)'', un patrón de diseño que implica construir una lista de
tuplas, ordenarlas, y extraer parte del resultado. \index{patrón DSU}
\item[reunir]
La operación de tratar una secuencia como una lista de argumentos.
\index{reunir}
\item[hashable (dispersable)]
Un tipo que tiene una función de dispersión. Los tipos inmutables, como
enteros, flotantes y cadenas son dispersables (hashables); los tipos
mutables como listas y diccionarios no lo son. \index{hashable}
\item[dispersar]
La operación de tratar una secuencia como una lista de argumentos.
\index{dispersar}
\item[modelado (de una estructura de datos)]
Un resumen del tipo, tamaño, y composición de una estructura de datos.
\index{modelado}
\item[singleton]
Una lista (u otra secuencia) con un único elemento. \index{singleton}
\item[tupla]
Una secuencia inmutable de elementos. \index{tupla}
\item[asignación por tuplas]
Una asignación con una secuencia en el lado derecho y una tupla de
variables en el izquierdo. El lado derecho es evaluado y luego sus
elementos son asignados a las variables en el lado izquierdo.
\index{asignación por tuplas} \index{tupla!asignación}
\end{description}




